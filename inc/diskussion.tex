Diskussion kring företagets verksamhet och dess arbetsätt. Jämförelser kring vad teorin säger och hur verkligeheten ser ut för MacGregor visar på intressanta skillnader samt hur dessa är gynnsamma för företaget.
%Jämför tidsplan med vad som faktiskt hände. 

% Egna tankar om jämförelserna som gjordes?
% Fanns det något som var överraskande/oväntat?
% Vad tycker vi om hur företaget jobbar?
% Vad har vi lärt oss?

% %
% %   JÄMFÖRELSE
% %

\subsection{Jämförelse mellan Teori och Verklighet}
Eftersom MacGregor har mer aktiviteter än teorin från \cite{dne} ger det möjligheten till högre noggrannhet i var resurser behövs och var besparingar kan göras men kräver också mer resurser för att genomföras. Genom att ha dedikerade säljare som jobbar med ABC-kalkyler inom företaget ger det Daniel en möjlighet att sammanställa all data på ett strukturerat sätt så man kan göra det bästa beslutet för produkten. Detta är essentiellt för företagets lönsamhet och vidare utveckling. Utan denna information som Daniel ger företaget så skulle inte några beslut kunna tas med god säkerhet om positivt resultat.

Tidsrapporteringssystemet som MacGregor använder sig av är av stor betydelse för uppbygget av en ABC-kalkyl och därför krävs det ärlighet inom företaget. Eftersom kalkylerna baseras på information från många delar av företaget är de också beroende på riktigheten av den informationen. Företaget måste då lita på att sina anställda rapporterar rätt och riktigt för att upprätta en bra kalkyl. Utbildning inom systemet är då nödvändigt för att anställda inte ska under eller överskatta tiden de lägger på respektive aktiviteter.

% \begin{itemize}
%     \item Är produktkalkylen rättvisande för produkten de säljer?
%     \item Var det som vi förväntade oss?(Jämför bakgrund och resultat, alltså hur det generellt fungerar och hur företaget faktiskt jobbar med det.)
% \end{itemize}

% %
% %   ARBETSSÄTT
% %

\subsection{Arbetsätt}
En ABC-kalkyl ger MacGregor den precisionen som behövs för att uppnå bäst resultat inom företaget. Detta tack vara att kalkylen gör det lätt att se var resurser behövs mest inom de olika aktiviteterna. Genom att noggrant se över var förbättringar och besparingar inom företaget kan göras, maximeras vinsten samt att utvecklingsmöjligheter klargörs. En ABC-kalkyl är däremot resurskrävande vilket gör att ekonomer inom företaget får en stor betydelse i vilket resultat företaget kommer gå med. Det löper också en risk att för mycket resurser läggs på denna typ av kalkyl. Nackdelarna är dock små jämfört med vilka fördelar som en välarbetad ABC-kalkyl ger inom ett företag som har resurser till detta. 

% Olika valutor -> Valutaeffekter -> Komplikationer för daniel(ekonomerna) men det blir lätt för företaget att fatta beslut när alla kalkyler framställs i koncernvalutan

% En stor fördel med koncernvaluta för MacGregor är att all produktkalkyl kan göras inom koncernvalutan vilket underlättar arbetet när man har fler valutor att hålla koll på. De personer som räknar på produktkalkyl behöver då inte bry sig om något annat när valutan lyfts bort från kalkylen. Ingenjörer gynnas enormt av detta då de slipper hantera valutaeffekterna som uppstår när man handlar med fler valutor. Om man istället inte skulle använda sig av en koncernvaluta så skulle det innebära att samma produkt kan ge olika resultat från år till år även om det säljs exakt lika stor kvantitet med samma kostnad. Detta är en god insikt att ha som en ingenjör när man jobbar för ett företag.
% Detta tycker vi är positivt med arbetssättet på MacGregor då det underlättar enormt för ingenjörer och skapar god lönsamhet för företaget genom att värna om samarbetet mellan ingenjörer och ekonomer.
% Samarbete inom ett företag är en nyckel till lönsamhet, god arbetsmiljö och effektivitet. 

% \begin{itemize}
%     \item Är företagets sätt att hantera deras produktkalkyl något bra?
% \end{itemize}

% %
% %   EGNA REFLEKTIONER
% %

\subsection{Reflektioner}
Generellt sätt tycker vi att MacGregor har ett effektivt sätt att arbeta på inom företaget som vi anser är väldigt optimerat för deras syfte. Genom detta arbete har vi fått erfarenhet om hur ett företag arbetar praktiskt med ekonomistyrningen produktkalkyl och hur denna utformas kring olika aktiviteter genom en ABC-kalkyl. Vi har också fått insikt i hur teori används samt anpassas till verkligheten för att uppnå bäst resultat inom företaget. Det är inte alltid teorin som har svaren utan många variabler, specifikt för företaget, som måste reflekteras över och teorin måste anpassas därefter.

% %
% %   FÖRBÄTTRINGAR
% %

\subsection{Förbättringar}


% MacGregors arbetssätt kretsar mycket, som vi ser det, kring att kalkylerna framläggs och man tar beslut utifrån vad dessa säger. Mycket kretsar då direkt på Daniels förmåga att sammanställa och leverera dessa inom en god tidsram. Som vi förstår från Daniel är detta en stor påfrestning på honom även fast han har hjälp. 
% Om resurser finns så skulle vi rekommendera att anställa ytterligare en hjälpreda, möjligtvis en assistent, till Daniel som kan stödja honom i det arbete han utför. Detta skulle innebära mer effektivitet och driva på utvecklingen av företaget genom fler noggrant utförda kalkyler som det kan tas beslut kring.

% \begin{itemize}
%     \item Finns det något som kan göras bättre? (som vi kan se)
%     \item Anställa ytterligare en som Daniel?
% \end{itemize}