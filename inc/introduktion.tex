För att erhålla insikter lämpliga för en ingenjör om ekonomistyrning har produktkalkylens betydelse och användning inom ett företag som tillämpar kalkylen dagligen undersökts.
Ett företag har kontaktats för intervju för att få insikt i hur det fungerar hos dom.
Det finns även andra typer av styrmedel, men vid inledande kontakt med företaget föreslogs det från företagets sida att man kunde undersöka produktkalkyl då stor kompetens inom just det området fanns.
Teori har lagts sida vid sida med hur det fungerar i praktiken och jämförts.

Förmågan att planera, utföra och slutrapportera ett grupparbete är även det något som är viktigt för en ingenjör.
Projektet ska bidra till styrkt erfaranhet inom detta område.

%
%   BAKGRUND
%

\subsection{Bakgrund} 
Ett företag behöver generellt sett ha en budget.
För att kontrollera budgeten används något som kallas för ekonomiska styrmedel.
En form av styrmedel kallas för självkostnadskalkyl.
För att få en uppfattning om företagets ekonomiska situation samt göra kloka beslut inför framtiden används diverse olika kalkyler.
En självkostnadskalkyl är den övergripande termen som beskriver metoden att fördela ett företags kostnader på godtyckliga kostnadsbärare.
Resultatet från en självkostnadskalkyl kan sedan användas som styrmedel för företagets ekonomiska framtid.
Ett exempel av en självkostnadskalkyl är produktkalkyl.
I en produktkalkyl är kostnadsbäraren företagets produkter och alla kostnader fördelas på dessa.
Varianter på produktkalkyler inkluderar påläggskalkyl, bidragskalkyl samt ABC-kalkyl. \cite{dne}

Företag, som har en försäljning av varor, får en stor fördel av att använda en produktkalkyl eftersom man snabbt kan se om en produkt är lönsam eller inte.
Eftersom civilingenjörer ofta är med och tar fram och/eller utvecklar nya produkter är denna form av kalkyl en av de första ingenjörer kommer i kontakt med.
En grundläggande förståelse för dessa typer av kalkyler underlättar då kommunikation med ekonomiavdelningen på ett företag; vi som utbildar oss till civilingenjörer drar nytta av att fördjupa oss i denna kalkyl. 

Ett företag som tillämpar produktkalkyl, har stor vikt vid deras ingenjörer och som var villiga att ställa upp på en intervju var MacGregor. 
Vid inledande kontakt med företaget föreslogs det från företagets sida att man kunde undersöka produktkalkylen ABC-kalkyl då stor kompetens inom just det området fanns.


\subsection{Teori} %TODO: Ska vi kalla detta något annat?

%generell ABC-kalkyl och hur den arbetas med.

I en ABC-kalkyl fördelas inte företagets kostnader ut på de färdiga produkterna direkt utan kostnadsbärarna är aktiviter.
Namnet kommer från engelskans {\bf A}ctivity {\bf B}ased {\bf C}osting.
Aktiviteter kopplas sedan ihop med kalkylobjekten via kostnadsdrivare.
Kostnadsdrivaren används för att mäta hur mycket kalkylobjekten förbrukar företagets resurser; aktiviteter.
I en fullständig ABC-kalkyl kan även direkta kostnader finnas som påförs kalkylobjekten direkt.
Precisionen hos kalkylen ökar med andelen direkta kostnader som påförs kalkylobjekten.\cite{dne}

Förr i tiden fick den ekonomiansvarige som ställde upp kalkylen gå runt till företagets anställda och intervjua de anställda för att bilda en uppfattning om tidsåtgång, kostnadsdrivare och arbetsflöde.
Nu för tiden kan man även göra som MacGregor och använda ett digitalt tidsrapporteringssystem där tidsåtgången för olika aktivitet direkt loggas och kan användas av ekonomerna\cite{daniel}.

När företaget använt en kalkyl en tid kan en efterkalkyl upprättas där kalkylens precision visas. Felaktigheter kan utrönas och nästa kalkyl kan kalibreras för att bli ännu bättre och mer verklighetsvisande.


Exemplen i \cite{dne} håller alltid antalet aktiviteter under 10, motiveringen brukar lyda att annars blir det för komplext.
Dock nämns det att ett ökat antal aktiviteter medför högre precision.