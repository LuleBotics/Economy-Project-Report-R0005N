Teori för hur produktkalkylering fungerar och hur den används hämtades främst från föreläsningar på Luleå Tekniska Universitet.
Litteraturen ''Den nya ekonomistyrningen" \cite{dne} användes också.
För att kunna jämföra teori med hur det praktiskt fungerar kontaktades företaget MacGregor för en intervjuv.
Ekonomiansvarig på företaget, Daniel Johansson, svarade på frågor om hur företaget utförde sina kalkyler.
Dessa frågor kan hittas i bilaga \ref{apdx:fragor}.

%
%   MACGREGOR
%

\subsection{MacGregor}
Företaget som intervjuats är MacGregor.

''MacGregor is a family of innovators. By offering engineering solutions and services for handling marine cargoes and offshore loads we make the sea more accessible, safe and reliable for those whose livelihood depends on the changing conditions of the sea" \cite{macgregor}.

MacGregor är ett världsledande företag inom godshandtering till havs. 
Man tillverkar och utvecklar verkyg för att lättare, säkrare och mer pålitligt kunna frakta gods till havs.
MacGregor har sitt Cargor and Material Handling Systems kontor i Örnsköldsvik. 

%
%   TIDSPLAN
%

\subsection{Tidsplan}
Projektet har innehållit gruppmöten för planering, kontakt med företaget samt intervjuer via skype med företaget.
Datum för alla aktiviter finns i Tabell \ref{tbl:tidsplan}.
\begin{table}[ht]
    \centering
    \caption{Tidsplan med datum och aktivitet för projektet}
    \label{tbl:tidsplan}
    \begin{tabular}{lllll}
        \toprule
        Datum & Aktivitet \\
        \midrule
        31/1 & Inledande kontakt med MacGregor via mail    \\
        2/2 & Gruppmöte för planering av projekt    \\
        7/2 & MacGregor godkände samabete med projektet    \\
        9/2 & Gruppmöte för planering av intervju    \\
        10/2 & Första intervjun med MacGregor via Skype   \\
        15/2 & Gruppmöte för planering av rapportskrivande    \\
        16-17/2 & Eventuell andra intervjun med MacGregor via Skype   \\
        17/2 & Macgregor läser igenom repporten   \\
        20/2 & Inlämning av rapport   \\
        24/2 & Gruppmöte för planering av återkoppling av annan rappport    \\
        27/2 & Inlämning av återkoppling av annan rapport  \\
        6/3 & Gruppmöte för planering av seminarium    \\
        8/3 & Seminarium \\
        \bottomrule
    \end{tabular}
\end{table}