
%%%%%%%%%%%%%%%%%%%%%%%%%%%%%%%%%%
%              Yo                %
%%%%%%%%%%%%%%%%%%%%%%%%%%%%%%%%%%

Resultatet baseras på svaren till frågorna i bilaga \ref{apdx:fragor} givna i intervju med MacGregor \cite{daniel}.

%
%   INTERVJU 1
%

\subsection{Intervju 1}
\begin{itemize}
    \item Vad fick vi ut från intervju 1? Svar på frågor
\end{itemize}
Oj oj oj.
Vi fick dom bästa svaren!
Dom absolut bästa!
Och du ska veta, du ska veta, jag känner till bra svar!
När jag säger att vi fick bra svar så betyder det att vi fick bra svar!
Gör Afrika bra igen!

%
%   INTERVJU 2
%

%\subsection{Intervju 2}
%\begin{itemize}
%    \item Vad behövde kompleteras från intervju 1?
%\end{itemize}

%
%   NERSKRIVEN INTERVJU, JAJAJAJA~
%

% Metoder för produktkalkylering?
% Direkt kostnadspåslag. Till största del ABC-kalkyl.

% ABC används för att definiera kalkylmodeller. De definierar deras Komponentgrupper, kostnadsgrupper, kostnadsdrivande element.

% Kostnadsdrivande element: (ex)
% Design
% Commission
% Faktiska designtimmar
% Varför ABC?
% Ger bäst uppskattning/koll på kostnader. Går man bara på material eller andra modeller får man mindre precision, speciellt eftersom de är ett internationellt bolag som handlar i många olika valutor (valutarisker). Bäst kalkylsäkerhet med ABC.
% I vilka situationer?
% Samtliga offerter. Kommer någon och frågar på en kran så ---
% Utvecklingsprojekt. Tex: Möjligheten att göra något billigare. Verkar det rimligt osv.
% Investeringar
% Kalkylobject?
% Komponenter
% För och efter?
% Jag, det är ett måste. Inga produkter är identiska. Det finns alltid gissningar som göras innan, och man vill alltid veta vad resultatet är. Effekten av valutaförändringarna behövs kollas av. Ju mer noggran man är ju bättre blir det. Vi gör därför för, under, och efterkalkyler.
% Blir det stora skillnader?
% När jag började var det en jättestor distans. Numera ligger vi runt ett par procent. 2% anses vara en stor avvikelse. 10-20% förrut. Företaget är väldigt nödja med hur det ser ut idag.

% Åt vilket håll man än ligger så blir det ofta dumt. Ligger man för lågt kan man förlora ordrar, eller om man ligger för dyrt så kan det göra väldigt ont på sista raden i budgeten.
% Tre vanligaste kalkylsitationerna?
% Offerering, investeringar, utveckling.
% Hur ser produktkalkylerna ut i respektive situation?
% Offerering: En produkt som ska ut. Komponentgrupper som tidigare nämnt.
% Utveckling: Då blir det även (inte bara kostnad/utvecklingskostnad) projektledninging och annat runt det som beräknas. Projektledare, sales, marketing. Man ser över behov av utbildning. Produktionskostnader (interna). Man kan ej gruppera ihop saker på samma sätt. Detaljkalkyl på lägsta nivå för att kunna följa upp.
% Vilka kalkylposter ingår
% Som tidigare.

% Kan se speciellt ut för oss pga att vi tillverkar marina produkter med väldigt hårda krav. Det blir väldigt specifikt. Vi handlar i 3 valutor. Det blir alltid valutaeffekter (nämnt tidigare).
% Vad beror skillnader på?
% Se ovan.
% Vem/vilka upprättar produktkalkyler?
% Jag, men även utesäljare har verktyg för att med enkla medel kunna utföra offertkalkylunderlag. De har riktigt detaljerade komponentkalkyler. Tillverkningsunderlag finns tabulerat osv. De väljer vilken krantyp de ska sälja, så ger programmet det som behövs (ihopställt av mig). Då kan ett slutpris räknas fram per automatik.
% Hur ofta?
% Varje dag. (Flera gånger om dagen?). Tre dedikerade säljare som inte gör annat än att upprätta kalkyler. Efterkalkyler görs dagligen för uppföljning.
% Vilka datakällor används?
% Efterkalkylsdata, all känd offertdata, specialkomponenter osv. Teknisk data (tekniken måste sitta ihop med produkten). Ritningar. En kalkyl kan vara svår att få fullständig tills dess att man är i slutskedet på en produkt (då ritningar finns att tillgå).
% Vilka använder kalkylerna?
% Säljarna som ska sätta pris. Ekonomiavdelningen använder de för månadsuppföljning. Ekonomiavdelningen är jätteberoende av kalkylerna. Contract managers som följer kontrakt. Inköpare. Alla som påverkar projekten behöver ha del av dem. I kalkylen står det hur mycket tid som förväntas användas. Diciplinen i företaget styrs upp av kalkylen.
% Är användarna nöjda med hur kalkylen upprättas?
% (Säljarna anser att det är för dyrt, hehe) Generellt är alla jättenödja. Det finns precision. Ett system som gör det simpelt för slutanvändarna att använda. Den enda som får jobba mycket är jag som måste producera all data som ligger bakom.

% Mycket är modulariserat och det går att automatisera. Det är dock svårt att automatisera många av sakerna. Speciellt för material som ska förädlas.

% Det som behöver automatiseras kodas av mig.
% Senaste ändringarna som är gjorda i produktkalkylen?
% Iom internationellt företag så har koncernvalutan ändrats. Låter enkelt, men när man tittar på valutaeffekter och vlautasamspel så kan man se många mysko effekter. Inget som man kan rekommendera. Samspel av många olika valutor blir det många spännande effekter.
% Krav: Hela strukturen var tvungen att läggas om. Alla beräkningsunderlag var baserad på en valuta, men behövdes då ändras till att klara av fler valutor. Något som egentligen borde gjorts före. Vid ett datum är det en kurs, men med tiden ändras kursen. Det blir ett väldigt samspel som kan ge konstiga effekter.

% Svenskt bolag. Finsk koncern. Hela MacGregor ska gå med <valuta> (säger dom). Då får man rätta sig efter det.

% Konstig effekt ex: En budget på x antal timmar design med x kr. Ska räknas om till en valuta på ett visst datum. Men löpande rapportering sker efter dagskurs. “Du får använda 400 timmar”. Men efter alla pengar kunde gått åt efter bara 300 timmar då kursen som var införd det tidigare datumet och kursen som var löpande var helt olika. Designern tyckte han hade 100 timmar kvar, men 0 pengar kvar. “Vi borde ha 100 timmar kvar, men inga pengar pga valutaeffekter”
% (Kan även gå åt andra hållet)

% (Affärssystem som är med och spökar. Mellanvalutor fram och tillbaka)

% Inför säkerhetsmarginaler. Man anpassar sig till verkligheten. Är viktigt när man ska kolla långt fram i tiden.
% Hur länge har ABC använts?
% Sen verktyget utvecklats. Verktyget spottar i stort sätt ut en ABC.

% Vi tycker ABC fungerar rätt bra. Vi kan känna att den kanske är väldigt simplifierad. Kanske inte fullständig? (säger han)

% 24 rader med aktiviteter (med detalj). Detaljerna ger trygghet. Det är så företaget vill hålla det. Det ger tydliga besparingsmöjligheter. Vi kan tydligt se vart vi tappar bort pengar som vi tror vi kan undvika. Optimeringsprocessen underlättas av all detalj.

% Kontinuerlig tidsrapport kommer per automatik pga affärssystemet.

% Nuvarande system har använts i ca 4-5 år.

% Största fixen? Konstiga besparingseffekter. (DCR) var ett projekt för att genomföra besparingar. Gjordes skoningslöst. Tog ej hänsyn till om det kunde utföras inom rimlig tid eller om det var relevant för produkten i fråga. Numera används detta ej längre då det kunde slå väldigt fel.
